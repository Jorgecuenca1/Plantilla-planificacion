\documentclass[
11pt, % The default document font size, options: 10pt, 11pt, 12pt
codirector, % Uncomment to add a codirector to the title page
]{charter} 




% El títulos de la memoria, se usa en la carátula y se puede usar el cualquier lugar del documento con el comando \ttitle
\titulo{Pesaje al paso} 

% Nombre del posgrado, se usa en la carátula y se puede usar el cualquier lugar del documento con el comando \degreename
\posgrado{Carrera de Especialización en Inteligencia artificial} 
%\posgrado{Carrera de Especialización en Internet de las Cosas} 
%\posgrado{Carrera de Especialización en Intelegencia Artificial}
%\posgrado{Maestría en Sistemas Embebidos} 
%\posgrado{Maestría en Internet de las cosas}

% Tu nombre, se puede usar el cualquier lugar del documento con el comando \authorname
\autor{Ing. Esp. Jorge Cuenca} 

% El nombre del director y co-director, se puede usar el cualquier lugar del documento con el comando \supname y \cosupname y \pertesupname y \pertecosupname
\director{Nombre del Director}
\pertenenciaDirector{FIUBA} 
% FIXME:NO IMPLEMENTADO EL CODIRECTOR ni su pertenencia
\codirector{Codirector} % para que aparezca en la portada se debe descomentar la opción codirector en el documentclass
\pertenenciaCoDirector{FIUBA}

% Nombre del cliente, quien va a aprobar los resultados del proyecto, se puede usar con el comando \clientename y \empclientename
\cliente{Ing. Guillermo}
\empresaCliente{Emtech}

% Nombre y pertenencia de los jurados, se pueden usar el cualquier lugar del documento con el comando \jurunoname, \jurdosname y \jurtresname y \perteunoname, \pertedosname y \pertetresname.
\juradoUno{Nombre y Apellido (1)}
\pertenenciaJurUno{pertenencia (1)} 
\juradoDos{Nombre y Apellido (2)}
\pertenenciaJurDos{pertenencia (2)}
\juradoTres{Nombre y Apellido (3)}
\pertenenciaJurTres{pertenencia (3)}
 
\fechaINICIO{14 de marzo de 2023}		%Fecha de inicio de la cursada de GdP \fechaInicioName
\fechaFINALPlan{18 de julio de 2023} 	%Fecha de final de cursada de GdP
\fechaFINALTrabajo{15 de agosto de 2023}	%Fecha de defensa pública del trabajo final


\begin{document}

\maketitle
\thispagestyle{empty}
\pagebreak


\thispagestyle{empty}
{\setlength{\parskip}{0pt}
\tableofcontents{}
}
\pagebreak


\section*{Registros de cambios}
\label{sec:registro}


\begin{table}[ht]
\label{tab:registro}
\centering
\begin{tabularx}{\linewidth}{@{}|c|X|c|@{}}
\hline
\rowcolor[HTML]{C0C0C0} 
Revisión & \multicolumn{1}{c|}{\cellcolor[HTML]{C0C0C0}Detalles de los cambios realizados} & Fecha      \\ \hline
0      & Creación del documento                                 &\fechaInicioName \\ \hline
1      & Desarrollo hasta la sección 9                & 19 de marzo de 2023\\ \hline
2      & Desarrollo hasta la sección 12                & 27 de marzo de 2023\\ \hline
%2      & Se completa hasta el punto 7 inclusive
%		  Se puede agregar algo más \newline
%		  En distintas líneas \newline
%		  Así                                                    & dd/mm/aaaa \\ \hline
%3      & Se completa hasta el punto 11 inclusive                & dd/mm/aaaa \\ \hline
%4      & Se completa el plan	                                 & dd/mm/aaaa \\ \hline
\end{tabularx}
\end{table}

\pagebreak



\section*{Acta de constitución del proyecto}
\label{sec:acta}

\begin{flushright}
Buenos Aires, \fechaInicioName
\end{flushright}

\vspace{2cm}

Por medio de la presente se acuerda con el \authorname\hspace{1px} que su Trabajo Final de la \degreename\hspace{1px} se titulará ``\ttitle'', consistirá esencialmente en {la implementación de un algoritmo para obtener un resultado válido de peso de un animal que está circulando sobre la plataforma de pesaje}, y tendrá un presupuesto preliminar estimado de {477} h de trabajo y \$7.200.000 colombianos, con fecha de inicio \fechaInicioName\hspace{1px} y fecha de presentación pública \fechaFinalName.

Se adjunta a esta acta la planificación inicial.

\vfill

% Esta parte se construye sola con la información que hayan cargado en el preámbulo del documento y no debe modificarla
\begin{table}[ht]
\centering
\begin{tabular}{ccc}
\begin{tabular}[c]{@{}c@{}}Dr. Ing. Ariel Lutenberg \\ Director posgrado FIUBA\end{tabular} & \hspace{2cm} & \begin{tabular}[c]{@{}c@{}}\clientename \\ \empclientename \end{tabular} \vspace{2.5cm} \\ 
\multicolumn{3}{c}{\begin{tabular}[c]{@{}c@{}} \supname \\ Director del Trabajo Final\end{tabular}} \vspace{2.5cm} \\
%\begin{tabular}[c]{@{}c@{}}\jurunoname \\ Jurado del Trabajo Final\end{tabular}     &  & \begin{tabular}[c]{@{}c@{}}\jurdosname\\ Jurado del Trabajo Final\end{tabular}  \vspace{2.5cm}  \\
%\multicolumn{3}{c}{\begin{tabular}[c]{@{}c@{}} \jurtresname\\ Jurado del Trabajo Final\end{tabular}} \vspace{.5cm}                                                                     
\end{tabular}
\end{table}




\section{1. Descripción técnica-conceptual del proyecto a realizar}
\label{sec:descripcion}


 % El bloque "consigna" se usa para poner texto en rojo y dar una pequeña ayuda sobre cómo completar la sección. En cada entrega parcial deben eliminar los comandos begin y end del bloque consigna de las secciones que hayan completado.
Se elabora el proyecto con la necesidad de obtener pesajes de animales al paso de manera certera, con el fin de poder hacer el tracking de peso de cada animal de manera remota y con esa información tomar decisiones sobre el rodeo, analizar si están bien alimentados, si deben cambiar de destino, si tienen enfermedades ya que no suben de peso o pierden peso, etc.
Además esta balanza es parte de una plataforma en la cual se puede realimentar de la información brindada para ajustar el dosificador de alimentos en caso de que lo requiera.

Se quiere lograr obtener de manera confiable el peso de los animales, utilizando un algoritmo entrenado con datos reales que pueda procesar los registros de las celdas de carga y obtener un peso con buena exactitud.
Estos datos en el equipo final son enviados a una plataforma IoT que mediante una aplicación móvil brinda al usuario del sistema la información de peso de cada animal que transita por la balanza.

Análisis exploratorio de datos:
Convertir los datos recolectados a un formato adecuado para su análisis, utilizando los datos adquiridos hasta el momento. Hacer una inspección de los datos rotulados para determinar la cantidad de muestras útiles para entrenamiento y evaluación. Sobre los datos útiles, se analizará la variedad de las etiquetas (ground-truth de los pesos) y la representatividad del espacio muestral que habría una vez desplegado en campo. Se sacarán algunas métricas estadísticas buscando sesgos y varianzas.

Esto servirá para estimar el tipo de modelo para realizar la regresión y su capacidad.

Especificación de nuevos datos a recolectar para ampliar el dataset:
En función del estudio de la aplicación y de lo que se observe en el paso anterior y del o los modelos que se propongan, se debe realizar una especificación para la ampliación de las muestras. La especificación incluirá casos particulares de interés.

Conformación del dataset de entrenamiento y evaluación:
Definir un entorno de desarrollo para el modelo a entrenar. Construir el pipeline para la transformación de los datos crudos al dataset depurado, para entrenar y evaluar el modelo. Esto incluirá la etapa de pre-procesamiento destinada sólo a la limpieza de datos.

Definición del pre-procesamiento a realizar a las muestras:
Este pre-procesamiento es el que habría que implementar en el pipeline de inferencia (estimación del peso) por cada muestra que se quiera procesar. Este pre-procesamiento, a priori, se implementará como lo mínimo necesario que ayude a incrementar el desempeño del modelo. Tener en cuenta que es distinto al pre-procesamiento de limpieza mencionado en el punto anterior.
Contempla el modelado del pre-procesamiento, no su implementación en la
plataforma.

Modelos:
Se implementarán y evaluarán al menos dos modelos, uno de mayor capacidad (para estimación off-line) que demuestre la viabilidad de la estimación, y otro de menor capacidad en función de las capacidades de cómputo disponibles para la estimación en tiempo real (que degrade el desempeño de manera tolerable).

Cuantización de un modelo para despliegue sobre la plataforma:
Este punto aplicaría si se implementará el modelo en una plataforma que no utilice punto flotante (procesadores de baja capacidad de cómputo o FPGAs), o bien que no soporte las librerías utilizadas en los modelos. Podría no ser necesario en una primera instancia de desarrollo.

Despliegue:
Si la plataforma no soporta las librerías usadas para ML, o bien no tiene una capacidad de cómputo medianamente aceptable (punto flotante, matricial), este paso puede demandar mucho trabajo.

Análisis de resultados:
Este análisis se refiere a la evaluación del desempeño una vez desplegado el modelo.Incluiría sugerencias para mejoras de desempeño, ya sea modificando y re-entrenando los modelos, como adquisiciones de nuevas muestras. Comparación de los datos obtenidos cotejados con pesos estáticos realizados en campo.


\begin{itemize}
	\item Este proyecto es parte del programa de vinculación con empresas del posgrado, en este caso se tiene la oportunidad de trabajar con la empresa EMTECH.
	
\end{itemize}

En la Figura \ref{fig:diagBloques} se presenta el diagrama en bloques del sistema. Se observa el proceso general para llevar a cabo el desarrollo del proyecto.

\begin{figure}[htpb]
\centering 
\includegraphics[width=.5\textwidth]{./Figuras/Figura1.png}
\caption{Diagrama en bloques del sistema}
\label{fig:diagBloques}
\end{figure}

\vspace{25px}



\section{2. Identificación y análisis de los interesados}
\label{sec:interesados}

% este comando se debe borrar para la entrega, junto con la contraparte \end{consigna}{red} 
 



\begin{table}[ht]
%\caption{Identificación de los interesados}
%\label{tab:interesados}
\begin{tabularx}{\linewidth}{@{}|l|X|X|l|@{}}
\hline
\rowcolor[HTML]{C0C0C0} 
Rol           & Nombre y Apellido & Organización 	& Puesto 	\\ \hline
Cliente       & \clientename      &\empclientename	&        	\\ \hline
Responsable   & \authorname       & FIUBA        	& Alumno 	\\ \hline
Orientador    &     -	      & \pertesupname 	& Director Trabajo final \\ \hline
\end{tabularx}
\end{table}


 % este comando se debe borrar para la entrega, junto con la contraparte \begin{consigna}{red}



\section{3. Propósito del proyecto}
\label{sec:proposito}


El propósito de este proyecto es crear un algoritmo que sea capaz de obtener un resultado válido de peso de un animal que está circulando sobre la plataforma de pesaje. Dependiendo del resultado
del algoritmo se utilizará este valor de peso para ajustes de raciones, análisis de sanidad, etc.


\section{4. Alcance del proyecto}
\label{sec:alcance}


El alcance del proyecto consiste en llevar a cabo un análisis exploratorio de datos con el objetivo de estimar el tipo de modelo para realizar la regresión y su capacidad. En primer lugar, se llevará a cabo la conversión de los datos recolectados a un formato adecuado para su análisis, seguido de una inspección de los datos rotulados para determinar la cantidad de muestras útiles para entrenamiento y evaluación. A continuación, se analizará la variedad de las etiquetas (ground-truth de los pesos) y la representatividad del espacio muestral que habría una vez desplegado en campo, extrayendo algunas métricas estadísticas para buscar sesgos y varianzas.

Posteriormente, se especificarán nuevos datos a recolectar para ampliar el dataset, en función del estudio de la aplicación y de lo que se observe en el paso anterior y del o los modelos que se propongan, incluyendo casos particulares de interés. Se definirá un entorno de desarrollo para el modelo a entrenar y se construirá el pipeline para la transformación de los datos crudos al dataset depurado para entrenar y evaluar el modelo. Esto incluirá la etapa de pre-procesamiento destinada sólo a la limpieza de datos.

Se definirá el pre-procesamiento a realizar a las muestras, el cual se implementará en el pipeline de inferencia por cada muestra que se quiera procesar. Este pre-procesamiento, a priori, se implementará como lo mínimo necesario que ayude a incrementar el desempeño del modelo, contemplando el modelado del pre-procesamiento, no su implementación en la plataforma.

Se implementarán y evaluarán al menos dos modelos, uno de mayor capacidad (para estimación off-line) que demuestre la viabilidad de la estimación, y otro de menor capacidad en función de las capacidades de cómputo disponibles para la estimación en tiempo real (que degrade el desempeño de manera tolerable). Además, se llevará a cabo la cuantización de un modelo para despliegue sobre la plataforma, si es necesario.

Finalmente, se realizará el despliegue del modelo y se llevará a cabo el análisis de resultados, evaluando el desempeño una vez desplegado el modelo e incluyendo sugerencias para mejoras de desempeño, ya sea modificando y re-entrenando los modelos, como adquisiciones de nuevas muestras. También se realizará una comparación de los datos obtenidos cotejados con pesos estáticos realizados en campo.

La implementación del modelo en la tarjeta de desarrollo no se hara en producción en el sitio de la balanza.

Los datos en el equipo final no seran enviados a una plataforma IoT de EMTECH, que mediante una aplicación móvil brinda al usuario del sistema la información de peso de cada animal que transita por la balanza.

\section{5. Supuestos del proyecto}
\label{sec:supuestos}

Para el desarrollo del presente proyecto se supone que:

\begin{itemize}
	\item El algoritmo proporciona mediciones precisas y consistentes del peso de los animales, independientemente del tamaño, raza o especie del animal
	\item  El algoritmo es capaz de detectar y corregir errores y factores de confusión, como la influencia del viento o el movimiento de la plataforma de pesaje, para proporcionar mediciones precisas y confiables del peso de los animales.
	\item  El algoritmo es capaz de proporcionar mediciones de peso en tiempo real y en un formato legible para su uso en ajustes de raciones alimenticias y análisis de sanidad de los animales.
	\item Los resultados de las mediciones de peso pueden ser utilizados para tomar decisiones sobre la salud y el bienestar de los animales, incluyendo su nutrición y tratamiento médico.
	\item La cantidad y calidad de los datos recolectados tendrán un impacto significativo en la precisión y confiabilidad del modelo de regresión utilizado para estimar el peso de los animales.
\end{itemize}



\section{6. Requerimientos}
\label{sec:requerimientos}


\begin{enumerate}
	\item Requerimientos funcionales
		\begin{enumerate}
			\item El sistema debe convertir los datos recolectados a un formato adecuado 					  para su análisis.
			\item El sistema debe inspeccionar los datos rotulados para determinar la 						cantidad de muestras útiles para entrenamiento y 							evaluación.
			\item Analizar la variedad de las etiquetas (ground-truth de 					  los pesos) y la representatividad del espacio 								muestral.
			\item El sistema debe sacar algunas métricas estadísticas buscando sesgos y varianzas.
			\item Especificar nuevos datos a recolectar para ampliar el 						dataset.
			\item Definir un entorno de desarrollo para el modelo a 							entrenar.
			\item El sistema debe tener construido el pipeline para la transformación de los datos crudos al dataset depurado, para entrenar y evaluar el modelo.
			\item Definir el pre-procesamiento a realizar a las muestras.
			\item Implementar y evaluar al menos dos modelos.
			\item El sistema debe tener cuantizado un modelo para despliegue sobre la plataforma.
			\item El sistema debe tener un modelo en la plataforma.
			\item El sistema debe analizar los resultados obtenidos y sugerir mejoras.
		\end{enumerate}
	\item Requerimientos de documentación
		\begin{enumerate}
			\item Documentar el proceso de conversión de los datos.
			\item Documentar el proceso de inspección de los datos.
			\item Documentar el proceso de análisis de las etiquetas y de 						las métricas estadísticas.
			\item Documentar la especificación de nuevos datos a recolectar.
			\item Documentar el entorno de desarrollo para el modelo a entrenar.
			\item Documentar el pipeline para la transformación de los datos crudos al dataset depurado.
			\item Documentar el pre-procesamiento a realizar a las muestras.
			\item Documentar los modelos implementados y evaluados.
			\item Documentar la cuantización del modelo.
			\item Documentar el despliegue del modelo en la plataforma.
			\item Documentar el análisis de resultados y las sugerencias de mejora.
		\end{enumerate}
	\item Requerimiento de testing
		\begin{enumerate}
			\item Realizar pruebas para garantizar la calidad del dataset depurado. 
			\item Realizar pruebas para garantizar la calidad del pre-procesamiento. 
			\item Realizar pruebas para garantizar la calidad de los modelos implementados.
		\end{enumerate}
	\item Requerimientos interoperabilidad
			\begin{enumerate}
			\item Asegurarse de que el modelo sea compatible con la plataforma donde será desplegado.
			\item Asegurarse de que el modelo cuantizado sea compatible con la plataforma donde será desplegado.
		\end{enumerate}
		\item Requerimientos asociados a la especialización
		\begin{enumerate}
			\item Se debe realizar una memoria técnica.
			\item Se debe realizar un informe de avance.
			\item Se debe realizar  el trabajo final del proyecto.
		\end{enumerate}
\end{enumerate}




\section{7. Historias de usuarios (\textit{Product backlog})}
\label{sec:backlog}
\begin{enumerate}
\item El metodo de story points que se utilizo fue planing poker usando la serie de fibonacci(0, 1, 2, 3, 5, 8, 13, etc).

\item Como analista de datos quiero poder importar archivos CSV para convertirlos a un formato adecuado para su análisis, para poder utilizar los datos adquiridos hasta el momento. Ponderación: 3 story points.

Criterio de ponderación: La tarea de importar archivos CSV es una tarea relativamente simple, pero requiere cierta cantidad de trabajo y tiempo. Por lo tanto, la historia se califica con un número moderado de story points.

\item Como usuario final quiero poder ver un resumen estadístico de los datos, incluyendo métricas como la media, la desviación estándar y el rango, para poder tener una idea general de la distribución de los datos. Ponderación: 5 story points.

Criterio de ponderación: La tarea de implementar un resumen estadístico de los datos implica algunas tareas más complejas, como la implementación de algoritmos de cálculo de estadísticas y la presentación de los resultados en un formato fácil de leer. Por lo tanto, la historia se califica con un número moderado-alto de story points.

\item Como administrador del sistema quiero poder agregar nuevas fuentes de datos al sistema para ampliar el conjunto de datos disponibles para el análisis, para poder mejorar la precisión de los modelos. Ponderación: 8 story points.

Criterio de ponderación: La tarea de agregar nuevas fuentes de datos implica varias tareas complejas, como la integración de nuevos formatos de archivo y la validación de los datos para asegurarse de que sean adecuados para su uso en el sistema existente. Por lo tanto, la historia se califica con un número alto de story points.

\item Como investigador científico quiero poder realizar análisis de correlación para determinar si hay relaciones significativas entre las variables, para poder identificar patrones y tendencias en los datos. Ponderación: 13 story points.

Criterio de ponderación: La tarea de implementar un análisis de correlación es un proceso más complejo que implica la implementación de un algoritmo de correlación y la presentación de los resultados en un formato útil para el usuario. Por lo tanto, la historia se califica con un número alto de story points.
\end{enumerate}

\section{8. Entregables principales del proyecto}
\label{sec:entregables}



Los entregables del proyecto son:

\begin{itemize}
	\item Informe de análisis exploratorio de datos: Este informe contendría los resultados de la exploración de los datos recolectados, incluyendo una descripción del formato adecuado para su análisis, el número de muestras útiles para entrenamiento y evaluación, la variedad de las etiquetas y la representatividad del espacio muestral.
	\item Especificación de nuevos datos a recolectar: Este entregable incluiría una lista de casos particulares de interés para la ampliación de las muestras, basados en el estudio de la aplicación y lo que se observó en el análisis exploratorio de datos.
	\item Dataset de entrenamiento y evaluación: Este entregable consistiría en el conjunto de datos depurados y listos para ser utilizados en el entrenamiento y evaluación de los modelos. Además, incluiría el pipeline de transformación de los datos crudos al dataset depurado.
	\item Definición del pre-procesamiento a realizar a las muestras: Este entregable consistiría en la especificación del pre-procesamiento a realizar a cada muestra para incrementar el desempeño del modelo.
	\item Implementación y evaluación de los modelos: este entregable consistiría en la implementación y evaluación de al menos dos modelos, uno de mayor capacidad y otro de menor capacidad en función de las capacidades de cómputo disponibles. Se proporcionarían los resultados de la evaluación del desempeño de cada modelo.
	\item Cuantización de un modelo para despliegue sobre la plataforma: este entregable incluiría la cuantización de un modelo para ser desplegado en una plataforma que no utilice punto flotante o que no soporte las librerías utilizadas en los modelos.
	\item Despliegue del modelo: este entregable incluiría la implementación del modelo en la plataforma destinada para su uso y el informe de los resultados obtenidos en el despliegue.
	\item Informe final: este entregable sería un informe completo del proyecto, incluyendo una descripción detallada de cada uno de los entregables mencionados anteriormente, los resultados obtenidos y las recomendaciones para mejoras de desempeño, ya sea modificando y re-entrenando los modelos, como adquisiciones de nuevas muestras. Además, se incluiría una comparación de los datos obtenidos cotejados con pesos estáticos realizados en campo.
\end{itemize}

\section{9. Desglose del trabajo en tareas}
\label{sec:wbs}


\begin{enumerate}
\item Creación e implementación del modelo
	\begin{enumerate}
	\item Convertir los datos recolectados a un formato adecuado para su análisis. (13 hs)
	\item Inspeccionar los datos rotulados para determinar la cantidad de muestras útiles para entrenamiento y evaluación. (13 hs)
	\item Analizar la variedad de las etiquetas y la representatividad del espacio muestral para estimar el tipo de modelo para realizar la regresión y su capacidad. (14 hs)
	\item Especificar nuevos datos a recolectar para ampliar el dataset.(12 h)
	\item Definir un entorno de desarrollo para el modelo a entrenar.(14 h)
	\item Construir el pipeline para la transformación de los datos crudos al dataset depurado, para entrenar y evaluar el modelo.(16 h)
	\item Definir el pre-procesamiento a realizar a las muestras para incrementar el desempeño del modelo.(15 h)
	\item Implementar y evaluar al menos dos modelos, uno de mayor capacidad y otro de menor capacidad.(22 h)
	\item Cuantizar un modelo para despliegue sobre la plataforma(14 h)
	\item Desplegar el modelo en una plataforma que soporte las librerías utilizadas para ML y tenga una capacidad de cómputo medianamente aceptable.(30 h)
	\item Realizar el análisis de resultados y sugerir mejoras de desempeño, modificando y re-entrenando los modelos, o adquiriendo nuevas muestras.(20 h)
		
	\end{enumerate}
\item Documentación técnica
	\begin{enumerate}
	\item Documentar el proceso de transformación de los datos crudos al dataset depurado, incluyendo los detalles de pre-procesamiento y limpieza de datos.(14 h)
	\item Documentar la especificación para la ampliación de las muestras.(14 h)
	\item Documentar la implementación de los modelos, incluyendo detalles sobre los hiperparámetros y la arquitectura utilizada.(14 h)
	\item Documentar la cuantización del modelo(14 h)
	\item Documentar el proceso de despliegue del modelo en la plataforma, incluyendo detalles sobre las librerías y capacidades de cómputo utilizadas.(14 h)
	\item Documentar los resultados obtenidos y las sugerencias para mejoras de desempeño.(14 h)
	\end{enumerate}
\item Pruebas
	\begin{enumerate}
	\item Realizar las pruebas necesarias para verificar que todo este funcionando.(24 h)
	\item Diseño de casos de prueba (14 h).
	\item Preparación del entorno de prueba(11 h).
	\item Ejecución de pruebas(15 h).
	\item Reporte de resultados(14 h).
	\item Corrección de errores(12 h).
	
	\end{enumerate}
	\item Documentación asociada a la especialización
		\begin{enumerate}
			\item Elaboración de memoria técnica de trabajo final (40 h).
			\item Realizar un informe de avance (40 h).
			\item Realizar  el trabajo final del proyecto (40 h).
		\end{enumerate}
\end{enumerate}

Cantidad total de horas: (477 h)




\section{10. Diagrama de Activity On Node}
\label{sec:AoN}
	
\begin{enumerate}
	
	\item Actividad A = Creación e implementación del modelo con duración de 183 h.
	\item Actividad B = Documentación técnica con duración de 84 h.
	\item Actividad C = Pruebas con duración de 90 h.
	\item Actividad D = Documentación asociada a la especializacion con duración de 120 h.
	
	
	\end{enumerate}
	En la Figura 2 se presenta el diagrama de activity on node.
\begin{figure}[htpb]
\centering 
\includegraphics[width=.8\textwidth]{./Figuras/AoN.png}
\caption{Diagrama de \textit{Activity on Node}.}
\label{fig:AoN}
\end{figure}


\begin{table}[ht]
%\caption{Identificación de los interesados}
%\label{tab:interesados}
\begin{tabularx}{\linewidth}{@{}|l|X|X|l|@{}}
\hline
\rowcolor[HTML]{C0C0C0} 
Color           & Descripción\\ \hline
\cellcolor{red!20}      & Actividades críticas o semicríticas    \\ \hline
\cellcolor{green!20}  & Actividades no críticas	\\ \hline
\cellcolor{blue!20}   &  Inicio y fin	   \\ \hline
\end{tabularx}
\end{table}







\section{11. Diagrama de Gantt}
\label{sec:gantt}

En la Figura 3 se presenta la tabla de Gantt, y en la Figura 4 se presenta su diagrama de Gantt rotado.
\begin{figure}[htpb]
\centering 
\includegraphics[width=.99\linewidth]{./Figuras/tabla.png}
\caption{Tabla del Gantt}
\label{fig:diagGantt}
\end{figure}


\begin{landscape}
\begin{figure}[htpb]
\centering 
\includegraphics[width=.90\linewidth]{./Figuras/Gantt-2.png}
\caption{Diagrama de Gantt rotado}
\label{fig:diagGantt}
\end{figure}

\end{landscape}




\section{12. Presupuesto detallado del proyecto}
\label{sec:presupuesto}


\begin{table}[htpb]
\centering
\begin{tabularx}{\linewidth}{@{}|X|c|r|r|@{}}
\hline
\rowcolor[HTML]{C0C0C0} 
\multicolumn{4}{|c|}{\cellcolor[HTML]{C0C0C0}COSTOS DIRECTOS} \\ \hline
\rowcolor[HTML]{C0C0C0} 
Descripción &
  \multicolumn{1}{c|}{\cellcolor[HTML]{C0C0C0}Cantidad} &
  \multicolumn{1}{c|}{\cellcolor[HTML]{C0C0C0}Valor unitario} &
  \multicolumn{1}{c|}{\cellcolor[HTML]{C0C0C0}Valor total} \\ \hline
Tarjeta de desarrollo FPGA &
  \multicolumn{1}{c|}{1} & 
  \multicolumn{1}{c|}{500.000 (COP)} &
  \multicolumn{1}{c|}{500.000 (COP)} \\ \hline
Ingeniero de software&
  \multicolumn{1}{c|}{1} &
  \multicolumn{1}{c|}{6.000.000 (COP)} &
  \multicolumn{1}{c|}{6.000.000 (COP)} \\ \hline

\multicolumn{3}{|c|}{SUBTOTAL} &
  \multicolumn{1}{c|}{6.500.000 (COP)} \\ \hline
\rowcolor[HTML]{C0C0C0} 
\multicolumn{4}{|c|}{\cellcolor[HTML]{C0C0C0}COSTOS INDIRECTOS} \\ \hline
\rowcolor[HTML]{C0C0C0} 
Descripción &
  \multicolumn{1}{c|}{\cellcolor[HTML]{C0C0C0}Cantidad} &
  \multicolumn{1}{c|}{\cellcolor[HTML]{C0C0C0}Valor unitario} &
  \multicolumn{1}{c|}{\cellcolor[HTML]{C0C0C0}Valor total} \\ \hline
Alquiler computador FPGA &
  \multicolumn{1}{c|}{1} & 
  \multicolumn{1}{c|}{400.000 (COP)} &
  \multicolumn{1}{c|}{400.000 (COP)} \\ \hline
Luz&
  \multicolumn{1}{c|}{1} &
  \multicolumn{1}{c|}{100.000 (COP)} &
  \multicolumn{1}{c|}{100.000 (COP)} \\ \hline
Internet&
  \multicolumn{1}{c|}{1} &
  \multicolumn{1}{c|}{200.000 (COP)} &
  \multicolumn{1}{c|}{200.000 (COP)} \\ \hline
\multicolumn{3}{|c|}{SUBTOTAL} &
  \multicolumn{1}{c|}{700.000 (COP)} \\ \hline
\rowcolor[HTML]{C0C0C0}
\multicolumn{3}{|c|}{TOTAL} &
 \multicolumn{1}{c|}{7.200.000 (COP)} 
   \\ \hline
\end{tabularx}%
\end{table}


\section{13. Gestión de riesgos}
\label{sec:riesgos}

\begin{consigna}{red}
a) Identificación de los riesgos (al menos cinco) y estimación de sus consecuencias:
 
Riesgo 1: detallar el riesgo (riesgo es algo que si ocurre altera los planes previstos de forma negativa)
\begin{itemize}
	\item Severidad (S): mientras más severo, más alto es el número (usar números del 1 al 10).\\
	Justificar el motivo por el cual se asigna determinado número de severidad (S).
	\item Probabilidad de ocurrencia (O): mientras más probable, más alto es el número (usar del 1 al 10).\\
	Justificar el motivo por el cual se asigna determinado número de (O). 
\end{itemize}   

Riesgo 2:
\begin{itemize}
	\item Severidad (S): 
	\item Ocurrencia (O):
\end{itemize}

Riesgo 3:
\begin{itemize}
	\item Severidad (S): 
	\item Ocurrencia (O):
\end{itemize}


b) Tabla de gestión de riesgos:      (El RPN se calcula como RPN=SxO)

\begin{table}[htpb]
\centering
\begin{tabularx}{\linewidth}{@{}|X|c|c|c|c|c|c|@{}}
\hline
\rowcolor[HTML]{C0C0C0} 
Riesgo & S & O & RPN & S* & O* & RPN* \\ \hline
       &   &   &     &    &    &      \\ \hline
       &   &   &     &    &    &      \\ \hline
       &   &   &     &    &    &      \\ \hline
       &   &   &     &    &    &      \\ \hline
       &   &   &     &    &    &      \\ \hline
\end{tabularx}%
\end{table}

Criterio adoptado: 
Se tomarán medidas de mitigación en los riesgos cuyos números de RPN sean mayores a...

Nota: los valores marcados con (*) en la tabla corresponden luego de haber aplicado la mitigación.

c) Plan de mitigación de los riesgos que originalmente excedían el RPN máximo establecido:
 
Riesgo 1: plan de mitigación (si por el RPN fuera necesario elaborar un plan de mitigación).
  Nueva asignación de S y O, con su respectiva justificación:
  - Severidad (S): mientras más severo, más alto es el número (usar números del 1 al 10).
          Justificar el motivo por el cual se asigna determinado número de severidad (S).
  - Probabilidad de ocurrencia (O): mientras más probable, más alto es el número (usar del 1 al 10).
          Justificar el motivo por el cual se asigna determinado número de (O).

Riesgo 2: plan de mitigación (si por el RPN fuera necesario elaborar un plan de mitigación).
 
Riesgo 3: plan de mitigación (si por el RPN fuera necesario elaborar un plan de mitigación).

\end{consigna}


\section{14. Gestión de la calidad}
\label{sec:calidad}

\begin{consigna}{red}
Para cada uno de los requerimientos del proyecto indique:
\begin{itemize} 
\item Req \#1: copiar acá el requerimiento.

\begin{itemize}
	\item Verificación para confirmar si se cumplió con lo requerido antes de mostrar el sistema al cliente. Detallar 
	\item Validación con el cliente para confirmar que está de acuerdo en que se cumplió con lo requerido. Detallar  
\end{itemize}

\end{itemize}

Tener en cuenta que en este contexto se pueden mencionar simulaciones, cálculos, revisión de hojas de datos, consulta con expertos, mediciones, etc.  Las acciones de verificación suelen considerar al entregable como ``caja blanca'', es decir se conoce en profundidad su funcionamiento interno.  En cambio, las acciones de validación suelen considerar al entregable como ``caja negra'', es decir, que no se conocen los detalles de su funcionamiento interno.

\end{consigna}

\section{15. Procesos de cierre}    
\label{sec:cierre}

\begin{consigna}{red}
Establecer las pautas de trabajo para realizar una reunión final de evaluación del proyecto, tal que contemple las siguientes actividades:

\begin{itemize}
	\item Pautas de trabajo que se seguirán para analizar si se respetó el Plan de Proyecto original:
	 - Indicar quién se ocupará de hacer esto y cuál será el procedimiento a aplicar. 
	\item Identificación de las técnicas y procedimientos útiles e inútiles que se emplearon, y los problemas que surgieron y cómo se solucionaron:
	 - Indicar quién se ocupará de hacer esto y cuál será el procedimiento para dejar registro.
	\item Indicar quién organizará el acto de agradecimiento a todos los interesados, y en especial al equipo de trabajo y colaboradores:
	  - Indicar esto y quién financiará los gastos correspondientes.
\end{itemize}

\end{consigna}


\end{document}
